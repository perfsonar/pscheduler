%
% pScheduler Test Guide for Idle
%

\def\testname{idle}


% TODO: Add titlepage to the options.
\documentclass[10pt]{article}

\input pscheduler-tex.tex

\DRAFT

\title{pScheduler Test Guide: {\it \testname}}
\author{The perfSONAR Development Team}


\begin{document}
\maketitle


%
% INTRODUCTION
%

\section{Introduction}

The {\tt \testname} test does nothing for a length of time called the
{\it idle} period.  This test is intended for use in development and
testing and not for use in producion.

Tools implementing this test should strive to be idle as efficiently
as possible.

This document describes schema version {\tt 1}.

\subsection{Participants}

This test includes the following participants:

\begin{center}
\begin{tabular}{|c|c|}
\hline
{\bf Number} & {\bf Role} \\
\hline
{\tt 0} & Time Waster \\
\hline
\end{tabular}
\end{center}



%
% TEST SPECIFICATION
%

\section{Test Specification Format}

\subsection{Description}

The test specification consists of a single JSON object containing the
pairs below.  \seejson


\typeditem{schema}{SchemaVersion} Version of the specification format.

\typeditem{duration}{Duration} Duration of the idle period.

\typeditem{host}{Host} The host which should be idle.  Defaults to the local system.

\typeditem{parting-comment}{String} Optional text to be output or
logged by the test at the conslusion of the idle period.

\typeditem{starting-comment}{String} Optional text to be output or
logged by the test at the beginning of the idle period.


\subsection{Example}
\begin{lstlisting}[language=json]
{
    "schema": 1,
    "duration": "PT30S",
    "starting-comment": "Starting to do nothing",
    "parting-comment": "Finished doing nothing"
}
\end{lstlisting}


%
% RESULT FORMAT
%

\section{Result Format}

\subsection{Description}
The result consists of a single JSON object containing the pairs
below.  \seejson

\typeditem{schema}{SchemaVersion} Version of the result format.

\typeditem{duration}{Duration} Length of time spent idle.  This should
match the duration in the specification.


\subsection{Example}
\begin{lstlisting}[language=json]
{
    "schema": 1,
    "duration": "PT30S"
}
\end{lstlisting}



%
% LIMIT FORMAT
%

\section{Limit Format}

\subsection{Description}
Limits for this test consist of a single JSON object containing the
pairs below.  \seelimit

\typeditem{duration}{Limit/Duration} Range of allowed idle times.

\typeditem{starting-comment}{Limit/String} String allowed for the starting comment.

\typeditem{starting-comment}{Limit/String} String allowed for the parting comment.


\example
\begin{lstlisting}[language=json]
{
    "duration": {
        "description": "Short periods",
        "range": {
            "lower": "PT15S",
            "upper": "PT1M"
        },
        "invert": false
    },
    "starting-comment" : {
	"description": "No platypi",
	"match": {
	    "style": "regex",
	    "match": "platypus",
	    "invert": true
	},
	"fail-message": "Starting comment cannot contain 'platypus'",
    },
    "parting-comment" : {
     
	"description": "Require a vowel if not empty''
	"match": {
	    "style": "regex",
	    "match": "(^\$|[aeiou])"
	}
	"fail-message": "Parting comment must contain a vowel if not empty", 
    }
}

\end{lstlisting}


\end{document}
