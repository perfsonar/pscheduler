%
% Standard LaTeX Utilities for pScheduler Documents
%

\usepackage{dirtree}
\usepackage{fullpage}
\usepackage{helvet}
\usepackage{hyperref}
\usepackage{listings}
\usepackage{parskip}
\usepackage{underscore}


% DRAFT Watermark

\usepackage{draftwatermark}
\SetWatermarkText{}

\def\DRAFT{
  \SetWatermarkText{DRAFT}
  \SetWatermarkScale{1.25}
  \SetWatermarkLightness{0.9}
}

\def\NOTDOC{
  \usepackage{draftwatermark}
  \SetWatermarkText{NOT DOCUMENTATION}
  \SetWatermarkScale{0.45}
  \SetWatermarkLightness{0.9}
}



% Highlighting
%
% Normally, you'd use the color and soul packages, but they have an
% internal naming conflict that never got fixed.  See
% http://tex.stackexchange.com/a/48549.
\usepackage{xcolor}
\usepackage[normalem]{ulem} % use normalem to protect \emph
\newcommand\hl{\bgroup\markoverwith
  {\textcolor{yellow}{\rule[-.5ex]{2pt}{2.5ex}}}\ULon}


\renewcommand{\familydefault}{\sfdefault}

\font\tt=rm-lmtl10
\font\itt=rm-lmtlo10
\font\btt=rm-lmtk10
\font\bitt=rm-lmtko10




% Utilities

\def\command#1{{\btt #1}}
\def\calloutitem#1{{\bf #1:}}
\def\example{For example:}
\def\false{{\tt false}}
\def\headerfile#1{{\tt \textless #1\textgreater} }
\def\jsontype#1{\textbf{\textit{#1}} as defined in the {\it pScheduler JSON Style Guide and Type Dictionary}}
\def\maychange{{\bf NOTE:} This information is subject to change.}
\def\note#1{{\bf NOTE:} {#1}}
\def\rootcommand#1{{\tt \# \command{#1}}}
\def\root{{\tt root}}
\def\seejson{See the {\it pScheduler JSON Style Guide and Type
    Dictionary} for a description of the types shown in parentheses.}
\def\seelimit{See the {\it pScheduler JSON Style Guide and Type
    Dictionary} and {\it The pScheduler Limit System} for descriptions of the types shown in parentheses.}
\def\stdvalue#1{\typeditem{#1}{Standard Value}  See the {\it pScheduler Standard Value Dictionary} for details.}
\def\todo#1{\hl{{\bf TODO:} #1}}
\def\true{{\tt true}}
\def\typeditem#1#2{{\bf {\btt #1} (#2)} ---}
\def\url#1{{\tt #1}}
\def\usercommand#1{{\tt \% \command{#1}}}


% Based on http://tex.stackexchange.com/a/83100
\lstdefinelanguage{json}{
    basicstyle=\footnotesize\ttfamily,
    xleftmargin=0.5in,
    xrightmargin=0.5in,
    numbers=none,
    numberstyle=\scriptsize,
    stepnumber=1,
    numbersep=8pt,
    showstringspaces=false,
    breaklines=true,
    frame=single
}
