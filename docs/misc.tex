%
% Misc. Bits of Documentation.  NOT FOR PRODUCTION.
%

% TODO: Add titlepage to the options.
\documentclass[10pt]{article}

\usepackage{msc}
\input pscheduler-tex.tex

\title{Miscellaneous Document Parts}
\author{The perfSONAR Development Team}

\NOTDOC

\begin{document}
\maketitle
%\tableofcontents




% -----------------------------------------------------------------------------

This document contains miscellaneous bits of wisdom.  It is not
intended to be an actual document, just a place to store things that
will eventually be part of other documents.  Section numbers are
meaningless.



\section{Multi-Participant Message Exchanges}

\setmscvalues{large}

%
% NOTE:  This section requires \usepackage{msc}
%

\def\parts{
  \declinst{o}{Originator}{}
  \declinst{p0}{Participant 0}{Lead}
  \declinst{p1}{Participant 1}{}
  \declinst{pn}{Participant {\it n}}{}
}
\def\mn#1#2#3{\mess{#1}{#2}{#3}\nextlevel}
\def\dn#1#2#3{\mess*{#1}{#2}{#3}\nextlevel}
\def\act#1#2{\action{#1}{#2}\nextlevel[2]}

\def\allparts#1{\condition{#1}{p0,p1,pn}\nextlevel[2]}

%
% Overview
%

\begin{msc}{Overview}
\parts
\mn{Task}{o}{p0}

\act{Tool Determination}{p0}
\dn{Unable}{p0}{o}
\act{Task Establishment}{p0}
\dn{Unable}{p0}{o}
\mn{Task Identifier}{p0}{o}
\act{Run(s)}{p0}
\mn{Fetch Result}{o}{p0}
\mn{Result}{p0}{o}
\act{Archive}{p0}

\end{msc}


%
% Tool Determination
%
\begin{msc}{Tool Determination}
\parts

\mn{Test Spec}{p0}{p1}
\mn{Tool list}{p1}{p0}

\mn{Test Spec}{p0}{pn}
\mn{Tool list}{pn}{p0}

\act{Intersect tool lists}{p0}
\dn{Unable}{p0}{o}
\act{Select Tool}{p0}

\end{msc}


%
% Task Establishment
%
\begin{msc}{Task Establishment}
\parts

\act{Assign Task ID}{p0}

\mn{Task, Tool}{p0}{p1}

\mn{Task}{p0}{p1}
\mn{OK}{p1}{p0}

\mn{Task}{p0}{pn}
\mn{OK}{pn}{p0}

\mn{Task ID}{p0}{o}

\end{msc}





%
% Run Scheduling
%
\begin{msc}{Run Scheduling}
\parts

\mn{Tentative Run Time}{p0}{p1}
\mn{Proposal}{p1}{p0}

\mn{Tentative Run Time}{p0}{pn}
\mn{Proposal}{pn}{p0}
\act{Intersect Proposals}{p0}
\mn{Schedule Run}{p0}{p1}
\mn{OK}{p1}{p0}



\mn{Schedule Run}{p0}{pn}
\mn{OK}{pn}{p0}

\end{msc}



%
% Post-Run Results Merge
%
\begin{msc}{Run}
\parts

\allparts{Wait for scheduled run time}

\action{Run Tool}{p0}
\action{Run Tool}{p1}
\action{Run Tool}{pn}
\nextlevel[3]

\mn{Fetch Result}{p0}{p1}
\mn{Result}{p1}{p0}

\mn{Fetch Result}{p0}{pn}
\mn{Result}{pn}{p0}

\act{Merge Results}{p0}
\mn{Full Result}{p0}{p1}
\mn{Full Result}{p0}{pn}

\mn{Fetch Full}{o}{p0}
\mn{Full Result}{p0}{o}

\act{Archive}{p0}

\end{msc}



\section{Sample Tool List Return}

This is what is returned when given a task and asked to find the tools
which can run it.  Note that the list of tools is not in any
particular order.

\begin{lstlisting}[language=json,firstnumber=1]
{
    "schema": 1,

    "tools": [
        { "name": "foo", "preference": 5 },
        { "name": "bar", "preference": 0 }
        { "name": "baz", "preference": -1 }
    ]
}
\end{lstlisting}




\section{Sample Task Spec}

\begin{lstlisting}[language=json,firstnumber=1]
{
    "schema": 1,

    "schedule": {
        "start": "PT1M",
        "slip": "PT15S",
        "duration": "PT5S",
        "repeat": "PT1M",
        "until": "2016-03-14T06:30:00",
    },

    "test": {
        "type": "idle",
        "spec": {
            "schema": 1,
            "starting-comment": "Starting to sleep.",
            "parting-comment": "Done sleeping."
        }
    },

    "features": [
        { "name": "mdvpn", "data": { "vpn": "abc123" }    },
        { "name": "pie",   "data": { "factor": "3.1415" } }
    ],

    "archives": [
        { ...TODO... }
    ]
}
\end{lstlisting}


\section{Sample Tool Input}

\begin{lstlisting}[language=json,firstnumber=1]
{
    "schema": 1,

    "schedule": {
        "start": "2015-12-20T19:18:17",
        "duration": "PT5S",
    },

    "test": {
        "type": "idle",
        "spec": {
            "schema": 1,
            "starting-comment": "Starting to sleep.",
            "parting-comment": "Done sleeping."
        }
    }
    "participant": 0
}
\end{lstlisting}



\section{Sample Tool Output}

\begin{lstlisting}[language=json,firstnumber=1]
{
    "schema": 1,

    "succeeded": true,

    "result": { ... Test-specific results for this participant ... }

    "diags": " ... Diagnostic output from tool ... ",
    "error": " ... Error output from tool ... ",
}

\end{lstlisting}





\section{Sample Full Run Result}

\begin{lstlisting}[language=json,firstnumber=1]
{
    "schema": 1,

    "task": "04a72888-8310-47ae-a913-e240f78a2711",
    "run": "2d0db9af-4570-426b-856a-449dab93f7f8",

    "schedule": {
        "start": "2015-09-20T04:15:38",
        "duration": "PT25S",
	"slip": "PT7S"
    },

    "test": {
        "type": "idle",
        "spec": {
            "schema": 1,
            "starting-comment": "Starting to sleep.",
            "parting-comment": "Done sleeping."
        }
    },

    "tool": "snooze",

    "participants": [
        {
            "participant": "perfsonar1.example.org",
            "result": {
                "succeeded": true,
                "result": { ... Test Result for This Participant ... }
                "diags": " ... Tool Diagnostic Output ... ",
                "error": " ... Tool Error Output ... ",
            },
            "time": {
                "start": "2015-09-20T12:19:56.31415",
                "error": 50000000    // Est. nanoseconds from true time
            }
        }
    ],

    "result": {
        ... Full Merged Result ...
    }

}
\end{lstlisting}


\section{Specifying Task Start Times}

Specify starting times for tasks in one of the following formats,
where {\itt interval} is an ISO 8601 interval and {\itt timestamp} is
an ISO 8601 timestamp:

\begin{center}
\begin{tabular}{lll}
{\tt soonest} & Earliest possible time & {\tt soonest} \\
{\itt interval} & {\itt interval} from right now & {\tt PT5M} \\
{\tt @{\itt interval}} & The nearest even {\itt interval} from right now & {\tt @PT1H} \\
{\itt timestamp} & At absolute time {\itt timestamp} & {\tt 2016-04-09T12:38:05}
\end{tabular}
\end{center}

Tricks:

Schedule at the earliest time within an interval:  Start {\tt now} and set slip to the interval.

Schedule at the earliest time after an interval:  Start {\tt PT5M} and slip infinitely.

Note that both of these have side effects on repetitions.  See issue \#4 in GitHub.



\end{document}
