%
% pScheduler Archiver Guide for Syslog
%

\def\archivername{syslog}


% TODO: Add titlepage to the options.
\documentclass[10pt]{article}

\input pscheduler-tex.tex

\DRAFT

\title{pScheduler Archiver Guide: {\it \archivername}}
\author{The perfSONAR Development Team}


\begin{document}
\maketitle


%
% INTRODUCTION
%

\section{Introduction}

The {\tt \archivername} archiver sends a raw JSON result to syslog.

This document describes schema version {\tt 1}.


%
% ARCHIVER DATA
%

\section{Archiver Data}

The {\tt data} object in the archiver specification may contain any of
the following:

\typeditem{ident}{String} An optional identifier to be prepended to
log entries.

\typeditem{facility}{String} The name of the facility to use in the log entries.
Valid values are {\tt kern}, {\tt user}, {\tt mail}, {\tt daemon},
{\tt auth}, {\tt lpr}, {\tt news}, {\tt uucp}, {\tt cron}, {\tt
syslog} and {\tt local0} through {\tt local7}.  If not provided, the
system default will be used.

\typeditem{priority}{String} The name of the priority to use in the log entries.
Valid values are {\tt emerg}, {\tt alert}, {\tt crit}, {\tt err}, {\tt
warning}, {\tt notice}, {\tt info}, and {\tt debug}.  If not provided,
the system default will be used.

\example
\begin{lstlisting}[language=json]
{
    "ident": "lab-network",
    "facility": "local3",
    "priority": "info"
}
\end{lstlisting}

\end{document}
