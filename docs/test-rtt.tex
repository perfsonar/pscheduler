%
% pScheduler Test Guide for Round Trip Time
%

\def\testname{rtt}


% TODO: Add titlepage to the options.
\documentclass[10pt]{article}

\input pscheduler-tex.tex

\DRAFT

\title{pScheduler Test Guide: {\it \testname}}
\author{The perfSONAR Development Team}


\begin{document}
\maketitle


%
% INTRODUCTION
%

\section{Introduction}

The {\tt \testname} test measures the amount of time taken for a
packet to leave one host, reach another and be returned to the host
where it originated.

This document describes schema version {\tt 1}.

\subsection{Participants}

This test includes the following participants:

\begin{center}
\begin{tabular}{|c|c|}
\hline
{\bf Number} & {\bf Role} \\
\hline
{\tt 0} & Originator \\
\hline
\end{tabular}
\end{center}



%
% TEST SPECIFICATION
%

\section{Test Specification Format}

\subsection{Description}

The test specification consists of a single JSON object containing the
pairs below.  \seejson

\typeditem{schema}{Cardinal} The schema version of this specification.

\typeditem{count}{Cardinal} The number of measurements to take.

\typeditem{dest}{Host} The network address or hostname of the receiver.

\typeditem{flow-label}{CardinalZero} Set the flow label on
measurements made to IPv6 addresses.  This is a 20-bit number and
therefore must be less than {\tt 1048576}. If {\tt 0}, a random value
will be selected and used.

\typeditem{interval}{Duration} Time to wait between measurements.

\stdvalue{ip-version} If {\tt dest} provides clues about the type of
address it represents, that type will be used.  Otherwise, the default
will be {\tt 4}.

\typeditem{source}{Host} The network address or hostname of the
interface to use.

\typeditem{suppress-loopback}{Boolean} Suppress loopback of multicast packets
when sending to a multicast address.

\typeditem{tos}{IPTOS} Type of service to be used.  \todo{This needs to
  be defined in the dictionary.}

\typeditem{length}{Cardinal} The length of test packets sent, in
bytes.

\typeditem{ttl}{Cardinal} IP time-to-live value for sent packets.

\typeditem{deadline}{Duration} How long the measurement should run
regardless of how many packets have been sent or received.

\typeditem{timeout}{Duration} Amount of time to wait for a response
for each packet sent.

\typeditem{hostnames}{Boolean} Whether or not attempts should be made
to map IP addresses discovered during the test to hostnames.  The
default is {\tt True}.


\subsection{Example}
\begin{lstlisting}[language=json]
{
    "schema": 1,
    "dest": "somehost.example.org",
    "ip-version": 4,
    "count": 10,
    "interval": "PT2S",
    "timeout": "PT1S"
}
\end{lstlisting}



%
% RESULT FORMAT
%

\section{Result Format}

\subsection{Description}
The result consists of a single JSON object containing the pairs
below.  \seejson

\typeditem{succeeded}{Boolean} Whether or not the test was a success.

\typeditem{roundtrips}{Array} An array of round trip objects as
described below.  The array should contain one object per packet sent
in order (i.e., the third element should contain information about the
thrid packet sent/received).

\typeditem{loss}{Number} Amount of packet loss, as a fraction
(e.g. {\tt 0.375} represents $37.5\%$ packet loss).  This pair is
required and should be {\tt 1.0} if {\tt roundtrips} contains no
elements.

\typeditem{min}{Duration} If any packets made a successful round trip,
the shortest time observed.

\typeditem{max}{Duration} If any packets made a successful round trip,
the longest time observed.

\typeditem{mean}{Duration} If any packets made a successful round trip,
the mean of the times observed.

\typeditem{stddev}{Duration} If any packets made a successful round
trip, the population standard deviation of all times observed.




\subsection{Round Trip Format}

Each item in the {\tt packets} array of the result consists of a single
JSON object containing the pairs below.  \seejson

Note that all pairs are optional, and a packet for which there was no
response will be an empty object.

\typeditem{length}{Cardinal} The length of the returned packet in bytes.

\typeditem{ip}{IPAddress} The IP from which the packet was returned.

\typeditem{ttl}{ttl} Time-to-live value on the returned packet.

\typeditem{rtt}{Duration} The round-trip time between the originator
and the responding host.

\typeditem{error}{ICMPError} The ICMP error reported for this
measurement, if any.


\subsection{Example}
\begin{lstlisting}[language=json]
{
    "schema": 1,

    "succeeded": true,
    "roundtrips": [
        {
          "ip": "198.51.100.209",
          "hostname": "foo.example.com",
          "rtt": "PT0.0159S",
          "length": 64
        },
        { },
        {
          "ip": "192.0.2.1",
          "error": "net-unreachable"
        },
        {
          "ip": "198.51.100.209",
          "hostname": "foo.example.com",
          "rtt": "PT0.0167S",
          "length": 64
        }
    ],
    "loss": 0.5,
    "min": "PT0.0159S",
    "max": "PT0.0167S",
    "mean": "PT0.0163S",
    "stddev": "PT0.0004S"
}
\end{lstlisting}


\end{document}
