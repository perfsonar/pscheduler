%
% pScheduler Limit System
%

% TODO: Add titlepage to the options.
\documentclass[10pt,titlepage]{article}

\input pscheduler-tex.tex

\DRAFT

\title{The pScheduler Limit System}
\author{The perfSONAR Development Team}


\begin{document}
\maketitle
\tableofcontents


\todo{This document is in the larval stage.}

\section{Introduction}
\todo{Write this.}

\section{Limit Concepts}
\todo{Write this.}


\section{Limit Configuration}
The limits for pScheduler are configured using a 

{\tt /etc/pscheduler/limits.conf}

The file is a plain text containing a single JSON object.  The general
format of the file is shown below, and details on each of the pairs
will be outlined in later sections.

\begin{lstlisting}[language=json]
{
    "identifiers": { ... },
    "classifiers": { ... },
    "limits": { ... },
    "applications": { ... }
}
\end{lstlisting}



\subsection{Identification}

Before any limits can be applied, pScheduler must determine what
entity (or {\it requester}) is asking for the task to be run.  This is
accomplished by matching one the requester's attributes against a
number of conditions linked to an {\it identifier}.  If the conditions
match, the requester is considered having been identified by the
identifier.  

A requesting entity may match multiple identifiers.  For example, a
user who had authenticaed using a password from IP address {\tt
  10.10.1.5} might be identified as {\tt bob} and {\tt private-ips}.

Identifiers are listed in the JSON in this format:

\begin{lstlisting}[language=json]
"identifiers": {
    "identifier-1": { ... },
    "identifier-2": { ... },
    ...
    "identifier-n": { ... }
}
\end{lstlisting}

pScheduler currently supports identification by the means described
below.


\subsubsection{IP Address}

This identification matches the requester's IP address against a list
of addresses or CIDR blocks.  These may be specified in the {\tt
  identifiers} section of the limit file like this:

\begin{lstlisting}[language=json]
"private-ip": {
    "description": "Private IP blocks per RFCs 1918 and 4193",
    "type": "ip-cidr-list",
    "data": [ "10.0.0.0/8", "172.16.0.0/12", "192.168.0.0/16",
              "fd00::/8" ]
}
\end{lstlisting}

Alternately, the list of IP blocks may be retrieved using a URL by
specifying the following:

\begin{lstlisting}[language=json]
"bogons": {
    "description": "Bogon/Martian addresses",
    "type": "ip-cidr-list-url",
    "data": {
        "sources": [
            "http://www.team-cymru.org/Services/Bogons/fullbogons-ipv4.txt",
            "http://www.team-cymru.org/Services/Bogons/fullbogons-ipv6.txt"
        ],
        "update": "PT1D",
        "retry": "PT1H"
    }
}
\end{lstlisting}

The retrieved content should be MIME type {\tt text/plain} with one
IPv4 or IPv6 address or CIDR block per line.  Text beginning with a
{\tt \#} is removed as a comment and whitespace is ignored.

The {\tt update} and {\tt retry} pairs indicate how often the list
should be fetched and how often to re-attempt after a failure.


\subsubsection{Host Name}

This identification matches the fully-qualified domain name (FQDN)
reverse-resolved requester's IP against a list of host names with
optional wildcards.  For example, if {\tt 198.51.100.38} reverse
resolves to {\tt perfsonar.example.org}, that will match {\tt
  *.example.org}. in the example below.

Host name identifiers are included in the limit configuration in this
format:

\begin{lstlisting}[language=json]
"example-domains": {
    "description": "Hosts in the example domains",
    "type": "hostname",
    "data": [
        "*.example.com", "*.example.net",
        "*.example.org", "*.example.edu",
        "*.example"
    ]
}
\end{lstlisting}

\note{For security reasons, the implementation of this identification
  must verify that the reverse-resolved host name forward resolves to
  the original IP.  There is also the risk that an adversary could
  slow pScheduler down by making many requests and taking a long time
  to reverse-resolve the IP.}



\subsubsection{Authenticated User}

A user authenticated using a login and password or a token provided by
a trusted third party.

\todo{Need JSON for this and some idea how it's going to be
  supported.}




\subsection{Classification}

Once one or more indentities have been established, 

\subsection{Limits}

\subsection{Application}

In the final stage, limits are applied to the classifications.

Application can be {\it pass-one}, where at least one of the applied limits is required to pass or {\it pass-all}

Each application is assigned a score, and a test passing 

Scores can be positive or negative, 




\end{document}
