%
% pScheduler JSON Style Guide and Dictionary
%

% TODO: Add titlepage to the options.
\documentclass[10pt]{article}

\input pscheduler-tex.tex

\DRAFT

\title{pScheduler JSON Style Guide and Type Dictionary}
\author{The perfSONAR Development Team}


\begin{document}
\maketitle

%
% Utilities
%

\def\source#1#2{{\bf #1:}  \url{#2}}


%
% INTRODUCTION
%

\section{Introduction}
\todo{Write this, mention ECMA-404.}

Sources of reference for the standards referred to multiple times by
this document can be found in section \autoref{standards}, {\it
  \nameref{standards}}.


%
% JSON STYLE
%

\section{JSON Style}
\todo{Write this.}

\subsection{Quotation Marks}
All element names and other string values will be surrounded by double
quotation marks ({\tt "}...{\tt"}).

\example\ {\tt "spaghetti"} and {\tt "meatballs"}.


\subsection{Element Names}
Element names will be lowercase with words separated by hyphens.

\example\ {\tt this}, {\tt that} and {\tt the-other}.


%
% TYPE DICTIONARY
%

\section{Type Dictionary}

This section covers simple and compound types used in pScheduler.


\subsection{Boolean}
A boolean value as defined by ECMA 404.

\example\ {\tt true} and {\tt false}.  (Note that these are also
the only two valid values.)


\subsection{Duration}
A length of time as defined by the ISO 8601 standard.

\example\ {\tt P3D}, {\tt PT15M} and {\tt P3DT2H6M18S}.


\subsection{Email}
An electronic mail address as defined by RFC 5322, section 3.4.

\example\ {\tt alice@finance.example.com} and {\tt
  bob.smith@eng.example.org}.

\source{IETF}{https://tools.ietf.org/html/rfc5322}


\subsection{Geographic Position}
A string representing coordinates on the Earth in the format defined
by ISO 6709, Annex H.  The trailing slash mandated by the standard is
optional but should be accepted if provided.

\example\ {\tt +48.8577+002.295/} and {\tt +40.6894-074.0447/}.

\source{ISO}{http://www.iso.org/iso/home/store/catalogue_tc/catalogue_detail.htm?csnumber=39242}\\
\source{Wikipedia}{https://en.wikipedia.org/wiki/ISO_6709}



\subsection{Integer}
A subset of {\tt number} as defined by ECMA 404 with the restriction
that the value contain no decimal point or scientific notation.

\example\ {\tt 0}, {\tt 3713} and {\tt -6264}.


\subsection{Maintainer}

An object containing identifying and contact information about the
maintainer of a piece of pSchedler software.  It contains these
elements:

\typeditem{name}{String} The name of the maintainer.

\typeditem{email}{Email} The maintainer's email address.

\typeditem{href}{URL} A URL where more information on the sofware or
the maintainer's organization can be found.

\example
\begin{lstlisting}[language=json]
{
    "name": "Example Software Development Team",
    "email": "software@example.org",
    "href": "http://www.example.org/software"
}
\end{lstlisting}


\subsection{Number}
A {\tt number} as defined by ECMA 404.

\example\ {\tt 47}, {\tt -19.3} and {\tt 6.022e23}.


\subsection{String}
A {\tt string} as defined by ECMA 404.

\example\ {\tt "Hello, World!"} and {\tt "He said
  \textbackslash"hello\textbackslash" as he went by."}


\subsection{Timestamp}
A date and time specified as defined the ISO 8601 standard.

\example\ {\tt 2015-02-27} and {\tt 2015-08-31T09:00:00-05}.


\subsection{URL}
A Uniform Resource Locator, Uniform Resource Identifier or Uniform
Resource Name specified as defined by RFC 3986.

\example\ {\tt http://www.example.org} and {\tt http://www.perfsonar.net}.

\source{IETF}{https://tools.ietf.org/html/rfc3986}


\subsection{Version}
A software version in the standard {\tt major.minor.patch} format.

\example\ {\tt 3.2.7}




%
% STANDARDS
%

\section{Standards}\label{standards}

\subsection{ECMA 404}
ECMA 404 describes a lightweight, plain-text format for data
interchange based on a subset of the JavaScript language.

\source{ECMA}{http://www.ecma-international.org/publications/files/ECMA-ST/ECMA-404.pdf}\\
\source{JSON.Org}{http://json.org}


\subsection{ISO 8601}
ISO 8601 is an international standard for describing dates, times,
intervals and repetition.

\source{ISO}{http://www.iso.org/iso/home/store/catalogue_tc/catalogue_detail.htm?csnumber=40874}\\
\source{Wikipedia}{https://en.wikipedia.org/wiki/ISO_8601}

\end{document}
